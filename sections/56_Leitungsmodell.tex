\newcolumn
\section{Leitungsmodell}

\subsection{Leitungsgleichungen}
\includegraphics[width=0.98\columnwidth, align=c]{images/Leitungsgleichungen_1.png}

\subsubsection{Allgemeine Differential Gleichung}
 
    \vspace{0.15cm}
    $
    \boxed{\frac{\partial u}{\partial x} = -\left(R' + L' \frac{\partial}{\partial t}\right)  \cdot i}
    \quad
    \boxed{\frac{\partial i}{\partial x} = -\left(G' + C' \frac{\partial}{\partial t}\right)  \cdot u}
    $
    \vspace{0.15cm}

\subsubsection{Allgemeine Differential Gleichung für Wechselstrom}
    
    \vspace{0.15cm}
    $
    \boxed{\frac{\partial U}{\partial x} = -\left(R' \cdot I + j \omega L' \cdot I\right)}
    \quad
    \boxed{\frac{\partial I}{\partial x} = -\left(G' \cdot U + j \omega C' \cdot U\right)}
    $

\subsubsection{Weitere Gleichungen}

$
\boxed{\frac{d^2 U}{dx^2} = \left(R' + j \omega L'\right)\left(G' + j \omega C'\right) \cdot U}
$

$
\boxed{\frac{d^2 I}{dx^2} = \left(R' + j \omega L'\right)\left(G' + j \omega  C'\right) \cdot I}
$

\vspace{0.15cm}

Mit folgender Definition von $\gamma$ ergibt sich:

\vspace{0.15cm}

$
\boxed{\underline{\gamma} = \sqrt{(R' + j\omega L')(G' + j \omega C')} = \alpha + j\beta}
$

$
\boxed{\frac{d^2 U}{dx^2} = \underline{\gamma}^2 \cdot U}
$

$
\boxed{\frac{d^2 I}{dx^2} = \underline{\gamma}^2 \cdot I}
$


\subsection{Lösung der Leitungsgleichung}

$
\boxed{\underline{U}(x) = \underline{U}_a + \underline{U}_b = \underline{U}^+ \cdot e^{-\underline{\gamma} x} + \underline{U}^- \cdot e^{\underline{\gamma} x}}
$

$
\boxed{\underline{I}(x) = \underline{I}_a + \underline{I}_b = \underline{I}^+ \cdot e^{-\underline{\gamma} x} + \underline{I}^- \cdot e^{\underline{\gamma} x}}
$

$
\boxed{\underline{I}(x) = \frac{-1}{R' + j\omega L'} \cdot \frac{d\underline{U}}{dx}
= \sqrt{\frac{G' + j\omega C'}{R' + j\omega L'}} 
\cdot \left( \underline{U}^+ \cdot e^{-\underline{\gamma} x} - \underline{U}^- \cdot e^{\underline{\gamma} x} \right)}
$

$
\boxed{\underline{Z}_W = \sqrt{\frac{R' + j\omega L'}{G' + j\omega C'}} \quad \ldots \text{ Wellenimpedanz in } \Omega}
$

\subsubsection{Wenn die Spannung der Leitung bekannt ist}

$
\boxed{\underline{U}(x = 0) = \underline{U}_1 = \underline{U}^+ + \underline{U}^-}
$

$
\boxed{\underline{I}(x = 0) = \underline{I}_1 = \frac{1}{Z_W} \left( \underline{U}^+ - \underline{U}^- \right)}
$

\subsubsection{Lösen nach $U^+$ und $U^-$}

$
\boxed{\underline{U}^+ = \frac{\underline{U}_1 + Z_W \cdot \underline{I}_1}{2}}
$

$
\boxed{\underline{U}^- = \frac{\underline{U}_1 - Z_W \cdot \underline{I}_1}{2}}
$

$
\boxed{\underline{U}(x) = \underline{U}_1 \cdot \frac{e^{\underline{\gamma} x} + e^{-\underline{\gamma} x}}{2} 
- Z_W \cdot \underline{I}_1 \cdot \frac{e^{\underline{\gamma} x} - e^{-\underline{\gamma} x}}{2}}
$

$
\boxed{\underline{U}(x) = \underline{U}_1 \cdot \cosh(\underline{\gamma} x) 
- Z_W \cdot \underline{I}_1 \cdot \sinh(\underline{\gamma} x)}
$

$
\boxed{\underline{I}(x) = \underline{I}_1 \cdot \cosh(\underline{\gamma} x) 
- \frac{\underline{U}_1}{Z_W} \cdot \sinh(\underline{\gamma} x)}
$


\newcolumn
\subsection{Allgemein und für 50Hz}
\subsubsection{Modell Allgemein mit exakter Zweitor-Gleichung}

\includegraphics[width=0.98\columnwidth, align=c]{images/Leitungsgleichungen_1.png}

\vspace{0.15cm}

$\boxed{
\begin{pmatrix}
    \underline{U}_1 \\
    \underline{I}_1
    \end{pmatrix}
    =
    \begin{pmatrix}
    \cosh(\underline{\gamma} \cdot l) & \underline{Z}_W \cdot \sinh(\underline{\gamma} \cdot l) \\
    \frac{1}{\underline{Z}_W} \cdot \sinh(\underline{\gamma} \cdot l) & \cosh(\underline{\gamma} \cdot l)
    \end{pmatrix}
    \begin{pmatrix}
    \underline{U}_2 \\
    \underline{I}_2
\end{pmatrix}}
$

$
\boxed{\underline{\gamma} = \sqrt{(R' + j\omega L')(G' + j \omega C')} = \alpha + j\beta}
$

\textbf{Leerlauf:} \quad $I_2 = 0 \Rightarrow U_2 = \frac{U_1}{\cosh(\gamma l)}$

\textbf{Kurzschluss:} \quad $U_2 = 0 \Rightarrow I_2 = \frac{I_1}{\cosh(\gamma l)}$





\subsection{PI-Ersatzschaltung (Modell Vereinfacht)}
Wenn $| \underline{\gamma } \cdot l | \ll 1$, kann die Vereinfachung $\cosh{\underline{\gamma l}} \approx 1$ und $\sinh{\underline{\gamma}l} = \underline{\gamma}l$:\\
Vereinfachung für „kurze“ Leitungen mit konzentrierten Elementen R, G, L, C
\begin{itemize}
    \item Querelemente bestehen aus Leitwert und Kapazität
    \item Längselemente bestehen aus Widerstand und Induktivität
    \item 50-Hz-Freileitungen bis ca. 250 km
    \item 50-Hz-Kabel bis ca. 50 km
\end{itemize}
\vspace{0.15cm}
\includegraphics[width=0.98\columnwidth, align=c]{images/Leitungsmodell_gültigkeit.png}
\vspace{0.15cm}
\begin{center}
    \begin{minipage}[t]{0.58\columnwidth}
       \centering
            $
            \boxed{
                \underline{Z} = (R' + j \cdot \omega \cdot L') \cdot l
            }
            \quad
            \boxed{
                \dfrac{\underline{Y}}{2} = \dfrac{(G' + j \cdot \omega \cdot C') \cdot l}{2}
            }
            $
    
    \end{minipage}\\ 
\end{center}




\vspace{0.15cm}
\includegraphics[width=0.98\columnwidth, align=c]{images/Leitungsgleichungen_2.png}
$\boxed{
\begin{pmatrix}
    \underline{U}_1 \\
    \underline{I}_1
    \end{pmatrix}
    =
    \begin{pmatrix}
    1 + \underline{Z} \cdot \dfrac{\underline{Y}}{2} & \underline{Z} \\
    \dfrac{\underline{Y}}{2} \cdot \left( 2 + \underline{Z} \cdot \dfrac{\underline{Y}}{2} \right) & 1 + \underline{Z} \cdot \dfrac{\underline{Y}}{2}
    \end{pmatrix}
    \begin{pmatrix}
    U_2 \\
    \underline{I}_2
\end{pmatrix}
}$\\
$
\boxed{
|U_2| = |U_1| \cdot 
\left| 
\frac{
\displaystyle \frac{1}{\left( \frac{G'}{2} + j \omega \frac{C'}{2} \right) \cdot l}
}{
(R' + j \omega L') \cdot l + 
\displaystyle \frac{1}{\left( \frac{G'}{2} + j \omega \frac{C'}{2} \right) \cdot l}
}
\right|
}$

\subsubsection{Übung Kraftwerk mit Doppelleitung an Netz}

\begin{minipage}[t]{0.6\columnwidth}
    \includegraphics[width=\columnwidth, align=c]{images/u03.png}
\end{minipage}
\hfill
\begin{minipage}[t]{0.58\columnwidth}
    $
        \boxed{
            \underline{U_1} = \underline{U_2} + j X'_L \cdot 
            \frac{P_{1,ph} - j Q_{1,ph}}{\underline{U_1}^*}
        }
    $
\end{minipage}




























