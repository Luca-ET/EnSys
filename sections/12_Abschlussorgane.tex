
\section{Abschlussorgane bei Speicherkraftwerken}

\subsection{Schützen}
\textbf{Aufgabe:}  
Binäre Regelung des Wasserdurchflusses („Auf und zu“)

\textbf{Anwendung:}  
Grundablass, Seeabschluss, Abschluss gegen das Unterwasser

\textbf{\textcolor{green}{Vorteile:}}  
Einfacher Aufbau, sichere Absperrung

\textbf{\textcolor{red}{Nachteile:}}  
Keine Zwischenstellungen („halb offen“ nicht möglich)

\textbf{Beispiel:}  
Regulierschütz und Reserveschütz beim Grundablass

\textbf{Sonstiges:}  
Zwei Schützen pro Grundablass für Betrieb und Revision; schnelle Schliess- und Öffnungszeiten sind entscheidend, um Druckstösse zu vermeiden.

\subsection{Klappen}
\textbf{Aufgabe:}  
Wasserfluss absperren (Notverschluss, Wartung)

\textbf{Anwendung:}  
Saugrohrklappen beim Kraftwerk Mapragg

\textbf{\textcolor{green}{Vorteile:}}  
Schnelles Schliessen, einfacher Aufbau

\textbf{\textcolor{red}{Nachteile:}}  
Keine Regelungsfunktion („halb offen“ nicht möglich)

\textbf{Beispiel:}  
Saugrohrklappen Mapragg

\textbf{Sonstiges:}  
Dienen als definitiver Abschluss bei Stillstand oder Wartung; Schliess- und Öffnungszeiten dürfen nicht verändert werden, um Druckstösse zu vermeiden.

\subsection{Drosselklappen}
\textbf{Aufgabe:}  
Regelbarer Durchfluss durch horizontales Verschieben (Spalt lässt Wasser durch)

\textbf{Anwendung:}  
Einsatz bei variablen Durchflüssen

\textbf{\textcolor{green}{Vorteile:}}  
Regelbarkeit des Durchflusses möglich

\textbf{\textcolor{red}{Nachteile:}}  
Höhere hydraulische Verluste

\textbf{Beispiel:}  
Horizontal verschiebbarer Rohrschieber

\textbf{Sonstiges:}  
Wird nur selten bei Speicher- oder Laufwasserkraftwerken eingesetzt, da dort meist binäre Regelung bevorzugt wird.

\subsection{Kugelschieber}
\textbf{Aufbau:}  
Drehbare Kugel mit Loch

\textbf{Aufgabe:}  
Absperren des Wasserflusses mit nahezu verlustfreiem Durchfluss im offenen Zustand

\textbf{Anwendung:}  
Definitiver Abschluss (Betriebsring und Reservering)

\textbf{\textcolor{green}{Vorteile:}}  
Im offenen Zustand nahezu verlustfrei

\textbf{\textcolor{red}{Nachteile:}}  
Wartung und Betrieb der Dichtung (Betriebsring) aufwändig

\textbf{Beispiel:}  
Kugelschieber in Wasserkraftwerken mit Betriebsring und Reservering

\textbf{Sonstiges:}  
Betriebsring sorgt für Abdichtung zwischen Gehäuse und Kugel, Reservering wird nur bei Revision geschlossen und gegen Wiederöffnen gesichert.

\subsection{Ring- und Eckringschieber}
\textbf{Aufgabe:}  
Steuerung des Wasserflusses innerhalb oder ausserhalb des Rohres

\textbf{Anwendung:}  
Ringschieber: innerhalb des Rohres; Eckringschieber: ausserhalb des Rohres

\textbf{\textcolor{green}{Vorteile:}}  
Flexibler Einbau, verschiedene Steuerungsmöglichkeiten

\textbf{\textcolor{red}{Nachteile:}}  
Erhöhter Konstruktions- und Wartungsaufwand

\textbf{Beispiel:}  
Ringschieber und Eckringschieber bei variablen Wasserführungen

\textbf{Sonstiges:}  
Ermöglichen Zwischenstellungen (feine Regelung), können jedoch zu höheren hydraulischen Verlusten führen.

