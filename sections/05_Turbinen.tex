\newpage
\section{Turbinen Kenngrössen}
\subsection{Nettofallhöhe und Durchfluss}

\subsection{Hydraulische Leistung}

$\boxed{P_{\text{hyd}} = \rho \cdot Q \cdot g \cdot H_n}$

\renewcommand{\arraystretch}{1.2} % Erhöht Zeilenhöhe für bessere Lesbarkeit
\begin{tabular}{@{} l p {7cm} l @{}}
    $[P_{\text{hyd}}]$  & Hydraulische Leistung \dotfill & $W$ \\
    $[Q]$               & Nutzwassermenge \dotfill & $m^3/s$ \\
    $[H_n]$             & Nettofallhöhe \dotfill & $m$ \\
    $[\rho]$            & Dichte des Wassers ($\rho = 1000$) \dotfill & $\frac{kg}{m^3}$ \\
    $[g]$               & Erdbeschleunigung ($g = 9.81$) \dotfill & $\frac{m}{s^2}$ \\
\end{tabular}




\subsection{Mechanische Leistung an der Turbinenwelle}

$\boxed{P_{\text{mech}} = \omega \cdot M}$

\renewcommand{\arraystretch}{1.2} % Erhöht Zeilenhöhe für bessere Lesbarkeit
\begin{tabular}{@{} l p {6cm} l @{}}
    $[P_{\text{mech}}]$  & Mechanische Leistung   \dotfill & $W$ \\
    $[\omega]$           & Winkelgeschwindigkeit \dotfill & $\frac{\text{rad}}{s}$ \\
    $[M]$                & Drehmoment            \dotfill & $Nm$ \\
\end{tabular}


\subsection{Winkelgeschwindigkeit}

$\boxed{\omega = 2 \cdot \pi \cdot n}$

\renewcommand{\arraystretch}{1.2} % Erhöht Zeilenhöhe für bessere Lesbarkeit
\begin{tabular}{@{} l p {6cm} l @{}}
    $[\omega]$  & Winkelgeschwindigkeit \dotfill & $\frac{\text{rad}}{s}$ \\
    $[n]$       & Drehzahl              \dotfill & $\frac{1}{s}$ \\
\end{tabular}


\subsection{Betriebszustände der Maschinengruppe}
\textbf{(Maschinengruppe = Turbine/Pumpe + Generator/Motor)}

\begin{itemize}
    \item Inselbetrieb
    \item Parallelbetrieb, Verbundbetrieb
    \item Instationäre Vorgänge
    \begin{itemize}
        \item Anfahren und Abstellen
        \item Lastabwurf $\Rightarrow$ Überdrehzahl
    \end{itemize}
\end{itemize}

\textbf{Durchgangsdrehzahl $n_D$} (auch Schleuderdrehzahl genannt) $\Rightarrow$ höchste erreichbare Drehzahl ohne Last (z.B. bei Versagen des Generators)

Die Durchgangsdrehzahl ist eine Bemessungsgröße. Die Maschinengruppe darf bei der Durchgangsdrehzahl keinen Schaden erleiden.



















