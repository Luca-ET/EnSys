\section{Atomkraftwerk}

\subsubsection{Merkmale Nukleare Dampferzeugung}

\begin{outline}
  \1 Leistungsfähige Energiequelle
  \1 CO2 - freie Produktion elektrischer Energie
  \1 Aufwändige Technologie
  \1 Sicherheit
  \1 Tiefenlager radioaktiver Stoffe
  \1 Diskussion in Politik, Gesellschaft, Ethik
\end{outline}


\subsubsection{Kernprozesse für die Energiegewinnung}

\begin{outline}
  \1 Künstliche Kernspaltung schwerer Kerne (Fission)
    \2 → Kernkraftwerke 3. Generation
    \2 (Stand der Technik)
    
  \1 Umwandlung von schweren Kernen in gut spaltbare Kerne im Brutprozess (Konversion)
    \2 → Kernkraftwerke 4. Generation
    \2 (in Entwicklung)
    
  \1 Verschmelzung leichter Kerne zu einem Kern (Fusion)
    \2 → Grundlagenforschung in Bearbeitung
\end{outline}


\subsection{Kernphysikalische Grundlagen}

$\boxed{A = Z + N}$ \quad Nuklid-Schreibweise: $\boxed{\text{\ce{^{A}_{Z}Element}} }$ \quad \text{z.\,B.} \quad \text{\ce{^{235}_{92}U}}

\vspace{1em}
\renewcommand{\arraystretch}{1.2}
\begin{tabular}{@{} l p{6cm} l @{}}
    $[A]$ & Anzahl Kerneteilchen eines Atoms \dotfill & $-$ \\
    $[Z]$ & Anzahl Protonen (Kernladungszahl) \dotfill & $-$ \\
    $[N]$ & Anzahl Neutronen \dotfill & $-$ \\
\end{tabular}



\subsection{Spaltung schwerer Kerne}

\begin{outline}
    \1 Spaltung schwerer Kerne
    \1 Einige Isotope besitzen die Eigenschaft, dass sie beim Beschießen mit langsamen Neutronen diese im Kern absorbieren und in zwei Tochterkerne zerfallen, wobei gleichzeitig 2–3 Neutronen frei werden.

    \vspace{0.15cm}

    $\boxed{
    \ce{^{235}_{92}U + ^{1}_{0}n \rightarrow  ^{89}_{36}Kr + ^{144}_{56}Ba + 3 ^{1}_{0}n + 200MeV}
    }$

    \vspace{0.15cm}

    \1 Bindungsenergie wird dabei frei.\\
        Im Mittel sind dies: 
        \textbf{200 MeV = $\bm{3{,}2 \cdot 10^{-11}}$ Ws pro Spaltung}
    \1 Die „schnellen“ Neutronen müssen abgebremst werden ($\Rightarrow$  thermische Neutronen), so dass der Prozess nicht abbricht.\\
    Dies geschieht mit einem Moderator wie „leichtes“ Wasser oder Graphit.
    \1 Werden genügend thermische Neutronen zur Verfügung gestellt, hält sich durch eine Kettenreaktion der Spaltungsprozess selbst aufrecht.
\end{outline}

\section{Gasturbinenkraftwerke}
Luft wird in einem Verdichter komprimiert. 
Die komprimierte Luft wird in einer Brennkammer mit Brennstoff (z.B. Erdgas, Heizöl) gemischt und verbrannt, wodurch heisse Gase entstehen. 
Diese heissen Gase expandieren in einer Turbine und treiben diese an. 
Die Turbine treibt wiederum den Verdichter und einen Generator zur Stromerzeugung an. \\
\textbf{Vorteile gegenüber Dampfturbinen-Kraftwerken}\\
\begin{itemize}
    \item Schnelle Startzeit und hohe Flexibilität: Gasturbinen können sehr schnell hoch- und heruntergefahren werden, was sie ideal für die Abdeckung von Spitzenlasten macht. 
    \item Geringerer Kühlwasserbedarf
    \item Kompaktere Bauweise
    \item Geringere Investitionskosten
    \item Geringere Emissionswerte (bei Erdgasbetrieb) im Vergleich zu konventionellen Kohle-Dampfkraftwerken. 
\end{itemize}

\subsection{Vor- und Nachteile}
\begin{itemize}
    \item \textcolor{green}{Vorteile:}
    \begin{itemize}
        \item Kompakter und einfacher Aufbau
        \item Hohe Leistungsdichte
        \item Schnelle Bereitschaft
        \item Relativ preisgünstige Investition
    \end{itemize}
    \item \textcolor{red}{Nachteile:}
    \begin{itemize}
        \item Nur hochwertige, schwefelarme Brennstoffe (Erdgas) sind wegen Schaufelkorrosion verwendbar.
        \item Begrenzter Wirkungsgrad
    \end{itemize}
\end{itemize}
\subsection{Funktionsprinzip}
\begin{minipage}[c]{1\columnwidth}
    $\boxed{du=c_v\cdot dT}$ Enthalpie bei isochorer (konstantes Volumen) Zustandsänderung\\
    $\boxed{dh=c_p\cdot dT}$ Enthalpie bei adiabatischer (Wärmedichte) Zustandsänderung\\
    $\boxed{R = c_p-c_v}$
    $\boxed{\kappa = \frac{c_p}{c_v}}$ 
    \begin{itemize}
        \item Einatomige Gase : $\kappa = 1.66$ (Helium, Neon, Argon, Krypton, Xenon, Radon)
        \item Zweiatomige Gase : $\kappa = 1.40$ (Moleküle mit zwei Atomen)
        \item Dreiatomige Gase: $\kappa = 1.30$ (Moleküle mit drei Atomen)
    \end{itemize}
    $\boxed{\Pi = \frac{p_2}{p_1} = \left( \frac{T_2}{T_1}\right)^{\frac{\kappa}{\kappa-1}}=\frac{p_3}{p_4}=\left( \frac{T_3}{T_4}\right)^{\frac{\kappa}{\kappa-1}}}$
    $\boxed{\frac{T_1}{T_2}=\frac{T_4}{T_3}}$\\
    $\boxed{\eta_\mathrm{th} = 1- \frac{Q_{\mathrm{ab}}}{Q_{\mathrm{zu}}} = 1-\frac{T_4-T_1}{T_3-T_2}=1-\frac{T_1}{T_2}=1-\frac{T_4}{T_3}=1-\Pi^{\frac{1-\kappa}{\kappa}}}$
    \begin{tabular}{@{} l p{6cm} l @{}}
        $[Q_{zu}]$ & Zugeführte Wärme \dotfill & $\mathrm{K}$ \\
        $[Q_{ab}]$ & Abgeführte Wärme \dotfill & $\mathrm{K}$ \\
        $[T_i]$ & Temperatur $i$ \dotfill &  $\mathrm{K}$ \\
        $[p_i]$ & Druck $i$ \dotfill &  $\mathrm{Pa = \frac{N}{m^2}}$ \\
        $[\eta_\mathrm{th}]$ & Wirkungsgrad \dotfill & $-$ \\
        $[\Pi]$ & Druckverhältnis \dotfill & $-$ \\
        $[R]$ & Spezifische Gaskonstante $8.314462$ \dotfill & $\mathrm{\frac{J}{mol\cdot K}}$ \\
        $[c_v]$ & Wärmekapazität (konstantes Volumen) \dotfill & $\mathrm{\frac{J}{kg\cdot K}}$ \\
        $[c_p]$ & Wärmekapazität (konstanter Druck) \dotfill & $\mathrm{\frac{J}{kg\cdot K}}$ \\
        $[\kappa]$ & Isotropen- bzw. Adiabatenexponent \dotfill & $-$ \\

        
        
    \end{tabular}
\end{minipage}
\includegraphics[width=1\linewidth]{images/FktprinzipGasturbine.png}

\subsection{Thermodynamische Grundlage}
\includegraphics[width=1\linewidth]{images/ThermodynamischeGrundlage.png}

\subsection{Gasturbine mit Luftvorwärmer}
Verbesserung des offenen GT - Prozesses\\
Vom Brennstoff aufzubringende Wärme verringert sich um $q_{WT}$, respektive die and die Umgebung abzuführende Wärme verringert sich um $q_{WT}$. Dadurch ergibt sich eine Verbesserung des therm. Wirkungsgrades.
\includegraphics[width=1\linewidth]{images/GasturbineLuftvorwärme.png}

\subsection{Gas- und Dampfturbinenkraftwerke (GuD)}
\begin{minipage}[c]{0.5\columnwidth}
    \includegraphics[width=1\linewidth]{images/GuD.png}
\end{minipage}
\begin{minipage}[c]{0.5\columnwidth}
    Eigenschaften:
    \begin{itemize}
        \item Gasturbine
        \begin{itemize}
            \item hohe Wärmeeintrittstemperatur (bis $1200\, \mathrm{^\circ C}$)
            \item hohe Wärmeaustrittstemperatur (ca. $500\, \mathrm{^\circ C}$) Dampfturbinen
        \end{itemize}
        \item Dampfturbine
        \begin{itemize}
            \item Relative niedrige Wärmeeintrittstem- peratur ($\le 550\, \mathrm{^\circ C}$)
            \item Niedrige Austrittstemperatur (ca. $40\, \mathrm{^\circ C}$)
        \end{itemize}
    \end{itemize}
\end{minipage}
Kombination beider Prozesse $\Rightarrow$ GuD mit hohem Wirkungsgrad
\begin{itemize}
    \item \textcolor{green}{Vorteile:}
    \begin{itemize}
        \item $5 - 10\, \mathrm{\%}$ höherer Wirkungsgrad als reine Dampf-KW
        \item Gutes Teillastverhalten (wichtig bei Lastfolgebetrieb, Einsatz Sekundärregulierung)
        \item Kostengünstiger Umbau älterer Dampf-KW
        \item Geringere Abwärme
    \end{itemize}
    \item \textcolor{red}{Nachteile:}
    \begin{itemize}
        \item $25 - 30\, \mathrm{\%}$ der zugeführten Wärme muss in Form eines hochwertigen Brennstoffes (Gas) erfolgen.
        \item Instandhaltungsaufwand an der hochbeanspruchten Gasturbine, Stillstandzeiten wegen der GT
    \end{itemize}
\end{itemize}




\subsubsection{Kombianlagen für Kraft-/Wärmekopplung (KWK)}
Thermische Kraftwerke (Wärme- Kraft- Prinzip) geben aufgrund physikalischer Randbedingungen einen grossen Teil der Wärme als Anergie ($60 - 70\, \mathrm{\%}$) an die Umgebung ab.\\
$\Rightarrow$ Mit der Nutzung der Abwärme kann der Gesamtwirkungsgrad wesentlich verbessert werden.

\begin{minipage}[c]{0.5\columnwidth}
\includegraphics[width=1\linewidth]{images/KwK.png}
\end{minipage}
\begin{minipage}[c]{0.5\columnwidth}
\begin{itemize}
    \item Abwärme wird für Prozess- oder Heizzwecke genutzt
    \item Gesamtwirkungsgrad bis $90\, \mathrm{\%}$
    \item Einen Freiheitsgrad: Strom und Wärme können nur gemeinsam erzeugt werden
    \item Zwei Freiheitsgrade: gekoppelt und ungekoppelt möglich
\end{itemize}
Wichtigste Anlagentypen:
\begin{itemize}
    \item Heizkraftwerk mit Gegendruckturbine
    \item Gasturbinenheizkraftwerk
    \item Heizkraftwerk mit Verbrennungsmotor (BHKW)
    \item Auskopplung von Prozessdampf hoher Temperatur
\end{itemize}
\end{minipage}
Wirkungsgrad:\\
\begin{minipage}[c]{0.2\columnwidth}
    $\boxed{\eta_{\mathrm{KWK}} = \frac{W_{\mathrm{el}}+Q_{\mathrm{H}}}{W_{\mathrm{BS}}}}$
\end{minipage}
\hfill
\begin{minipage}[c]{0.6\columnwidth}
    \begin{tabular}{@{} l p{4cm} l @{}}
        $[W_{\mathrm{el}}]$ & Stromproduktion \dotfill & $\mathrm{kWh}$ \\
        $[Q_{\mathrm{H}}]$  & Wärmeauskopplung \dotfill & $\mathrm{J}$ \\
        $[W_{\mathrm{BS}}]$ & Zugeführte Brennstoffenergie\dotfill & $\mathrm{J}$ \\
    \end{tabular}
\end{minipage}

\subsection{Solarthermische Kraftwerke}
Stromerzeugung wie bei thermischen Kraftwerken, jedoch mit Sonne
Arten:
\begin{itemize}
    \item Parabolrinnenkonzentrator
    \item Heliostatenfeld mit Turm
    \item Paraboloid-Dish
    \item Fresnel-Linse
\end{itemize}

\subsection{Geothermische Kraftwerke}
\begin{itemize}
    \item Temperatur im Innern der Erde: \text{$5000\text{–}6000\,^\circ\text{C}$} Wärmestrom zur Oberfläche (Abkühlung)
    \item Temperaturgradient: $3\, \mathrm{K}$ pro $100\, \mathrm{m}$
    \item Geothermische Anomalien → neben der Wärmeleitung noch Konvektion bzw. Wärmetransport durch Materialtransport (Aufstieg von glutflüssigem Magmas oder aufwärtsgerichtete Grundwasserbewegungen, aber meist in Erdbebengebiten)
    \item Umweltbeeinflussung:
    \begin{itemize}
        \item chemisch agressive und z.T. giftige Bestandteile ->Bauteiele müssen Korrisionsbeständig sein
        \item Zurückpumpen, verhindert auch das Absenken des Bodens
    \end{itemize}
\end{itemize}

\subsection{Verstromung von Biomasse}
Energiegewinnung aus Pflanzen oder Pflanzenresten
Verwendete Materialien:
\begin{itemize}
    \item eigens dafür kultivierte landwirtschaftliche Nutzpflanzen wie Mais oder Raps
    \item schnell wachsende Gehölze
    \item Abfall- und Reststoffe aus Landwirtschaft, Haushalten und Industrie (beispielsweise Hackschnitzel aus der
Holzindustrie, Altfett aus der Lebensmittelherstellung, aber auch Klärschlamm)

\end{itemize}


\newcolumn