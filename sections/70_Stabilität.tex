\newcolumn
\section{Stabilität}
Idealzustand:
\begin{itemize}
    \item Alle Generatoren drehen mit der gleichen konstanten Drehzahl
    \item Amplituden der Knotenspannungen sind konstant
    \item Polradwinkel / Spannungswinkel sind konstant
\end{itemize}
\subsection{Stabilitätsbegriff}

Die Stabilität des Stromnetzes ist die Fähigkeit eines elektrischen Stromnetzes, bei einem gegebenen Ausgangszustand \textbf{nach einer physikalischen Störung wieder einen Gleichgewichtszustand zu erreichen}, wobei die meisten Systemvariablen so begrenzt sind, dass praktisch das gesamte System intakt bleibt.


\subsection{Aufgaben des Netzbetriebs}

\begin{tabular}{|l l|}
\hline
\textbf{{Technische Ziele:}} & \textbf{{Zu jedem Zeitpunkt Sicherstellung von ...}} \\
\hline
\textbf{Leistungsbilanz:} & Angebot = Nachfrage \\
\textbf{Synchronizität und Dämpfung:} & gleiche Frequenz an allen Knoten \\
\textbf{Frequenzstabilität:} & Frequenz = 50 Hz \\
\textbf{Spannungsstabilität:} & Spannung = Nennspannung \\
\hline
\end{tabular}

\vspace{0.15cm}

\begin{tabular}{|l l|}
\hline
\textbf{Technische Herausforderungen:} & \textbf{Beschreibung} \\
\hline
\textbf{Variabilität} & Fluktuierende Last- und Angebotsverläufe \\
\textbf{Unsicherheit} & Ungenaue Last- und Angebotsvorhersagen \\
\textbf{Störungen} & Ausfälle, Leistungsoszillationen, Blackouts \\
\hline
\end{tabular}

\subsection{Statische Stabilität (Kleinsignalstabilität)}
\begin{itemize}
    \item Es kann sich ein stabiles Gleichgewicht einstellen (Eingeschwungener Zustand)
    \item Lastfluss bildet konstanter Frequenz, konstanten Spannungen und Winkeln aus
    \item Entscheidend ist die Konfiguration des Systems:
    \begin{itemize}
        \item Einspeisung
        \item Netz (Eigenschaften der Netzelemente \& Topologie)
        \item Lasten
    \end{itemize}
\end{itemize}

\subsection{Dynamische Stabilität (Grosssignalstabilität)}

\begin{itemize}
    \item Nach einer Störung stellt sich erneut ein stabiles Gleichgewicht ein
    \item Frequenz, Spannung und Winkel sind vorübergehend veränderlich
    \item Entscheidend ist die Konfiguration des Systems sowie
    \begin{itemize}
        \item Art der Störung (Schaltung, Kurzschluss, Lastsprung, ...)
        \item Dauer der Störung (Wiedereinschaltzeit)
    \end{itemize}
\end{itemize}