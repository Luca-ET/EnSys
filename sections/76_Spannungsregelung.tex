\section{Spannungsregelung}
Unter Spannungsstabilität versteht man die Fähigkeit eines Stromversorgungssystems, an allen Bussen (Knoten) im System konstante Spannungen aufrechtzuerhalten, nachdem es einer Störung aus einem gegebenen Betriebszustand ausgesetzt wurde.


\subsection{Spannungskollaps}

\begin{itemize}
    \item Spannung zu hoch: Isolationsdurchbruch
    \item Spannung zu tief: Spannungskollaps
    \item Abhängig von Nasenkurve
\end{itemize}

\vspace{0.15cm}

\includegraphics[width=0.98\columnwidth, align=c]{images/Spannungsregelung_1.jpg}

\vspace{0.15cm}

\includegraphics[width=0.68\columnwidth, align=c]{images/Spannungsregelung_2.jpg}

\vspace{0.15cm}

\begin{minipage}[c]{0.48\columnwidth}
    \begin{center}
        \includegraphics[width=0.98\columnwidth, align=c]{images/Spannungsregelung_3.jpg}
    \end{center}    
\end{minipage}
\hfill
\begin{minipage}[c]{0.48\columnwidth}
    \begin{center}
        \includegraphics[width=0.98\columnwidth, align=c]{images/Spannungsregelung_4.jpg}
    \end{center}
\end{minipage}

\vspace{0.3cm}

\begin{minipage}[c]{0.48\columnwidth}
    \begin{center}
        \includegraphics[width=0.98\columnwidth, align=c]{images/Spannungsregelung_5.jpg}
    \end{center}    
\end{minipage}
\hfill
\begin{minipage}[c]{0.48\columnwidth}
    \begin{center}
        \includegraphics[width=0.98\columnwidth, align=c]{images/Spannungsregelung_7.jpg}
    \end{center}
\end{minipage}

\vspace{0.15cm}

\begin{minipage}[c]{0.48\columnwidth}
    \begin{center}
        \includegraphics[width=0.98\columnwidth, align=c]{images/Spannungsregelung_6.jpg}
    \end{center}    
\end{minipage}
\hfill
\begin{minipage}[c]{0.48\columnwidth}
    \begin{center}
        \includegraphics[width=0.98\columnwidth, align=c]{images/Spannungsregelung_8.jpg}
    \end{center}
\end{minipage}



\subsection{Spannungs-Blindleistungs-Regelung}

\subsubsection{Spannungsänderung}

\begin{minipage}[c]{0.48\columnwidth}
    \begin{center}
        \includegraphics[width=0.98\columnwidth, align=c]{images/Spannungs-Blindleistungs-Regelung_1.png}
    \end{center}    
\end{minipage}
\hfill
\begin{minipage}[c]{0.48\columnwidth}
        \includegraphics[width=0.5\columnwidth, align=c]{images/Spannungs-Blindleistungs-Regelung_2.jpg}
\end{minipage}

\vspace{0.15cm}

$
\boxed{\Delta U_{\%} = \frac{\Delta Q}{S_K} \cdot 100\%} 
\quad
\boxed{\Delta U = \frac{\Delta Q}{S_K} \cdot U} 
$

\vspace{0.15cm}

\renewcommand{\arraystretch}{1.2}
\begin{tabular}{@{} l p{6cm} l @{}}
    $[S_K]$  & Kurzschlussleistung \dotfill               & $\text{VA}$ \\
    $[\Delta Q]$ & Veränderung der Blindleistung \dotfill & $\text{VAR}$ \\
    $[U]$ & Referenz-Spannung ($\mathrm{220 \ kV}$ oder $\mathrm{380 \ kV}$) \dotfill & $\mathrm{V}$ \\
    $[\Delta U]$ & Spannungsänderung \dotfill & $\mathrm{V}$ \\
    $[\Delta U_{\%}]$ & Spannungsänderung in Prozent \dotfill & $\%$ \\
\end{tabular}

Eine Kapazität speist kapazitive Blindleistung ein, was den Spannungsfall über der Netzimpedanz verringert bzw. kompensiert. Dadurch erhöht sich die Knotenspannung.

\subsubsection{Blindleistung}

\begin{itemize}
    \item Q-Bilanz eines Netzes ist immer ausgeglichen
    \item Q wirkt unmittelbar und lokal auf U
    \item Q fließt von höherer zu tieferer U
    \item Wenn Blindleistung aus dem Netz genommen wird sinkt die Spannung
\end{itemize}

\vspace{0.15cm}

\includegraphics[width=0.75\columnwidth, align=c]{images/Spannungs-Blindleistungs-Regelung_3.jpg}\\



\subsection{Grundlagen Spannungshaltung}
\includegraphics[width=0.98\columnwidth, align=c]{images/Grundlagen_Spannungshaltung_1.jpg}



\subsection{Spannunshaltung in der Schweiz}
\includegraphics[width=0.98\columnwidth, align=c]{images/Spannunshaltung_in_der_Schweiz_1.jpg}


\subsection{Kraftwerke}

\begin{itemize}
    \item geregelte Blindleistungsquellen
    \item Vergütung der anforderungskonformen Lieferung von Blindleistung.
\end{itemize}

\vspace{0.15cm}

\begin{minipage}[c]{0.48\columnwidth}
    \begin{center}
        \includegraphics[width=0.98\columnwidth, align=c]{images/Kraftwerke_1.jpg}
    \end{center}    
\end{minipage}
\hfill
\begin{minipage}[c]{0.48\columnwidth}
    \begin{center}
        \includegraphics[width=0.8\columnwidth, align=c]{images/Kraftwerke_2.jpg}
    \end{center} 
\end{minipage}


\subsection{Unterlagerte Netze}

\begin{minipage}[c]{0.38\columnwidth}
    \begin{center}
        \includegraphics[width=0.98\columnwidth, align=c]{images/Unterlagerte_Netze.jpg}
    \end{center}    
\end{minipage}
\hfill
\begin{minipage}[c]{0.58\columnwidth}
    \[
    \boxed{
        K = \frac{|\Delta U_m|}{|W_Q|} \leq \frac{1}{K_{\text{max}}}
    }
    \]
    \vspace{0.2cm}
    \renewcommand{\arraystretch}{1.2}
    \begin{tabular}{@{} l p{3cm} l @{}}
        $[W_P]$  & Wirkenergieaustausch     \dotfill    &   $\text{W}$ \\
        $[W_Q]$  & Blindenergieaustausch    \dotfill    &   $\text{var}$ \\
        $[LF]$   & Leistungsfaktor (0{,}9–1)   \dotfill    &   $\text{-}$ \\
        $[\Delta U_m]$ & Spannungsabweichung  \dotfill    &   $\text{kV}$ \\
        $[K]$    & Verhältnis Spannungsabweichung zu Blindenergie \dotfill & $\text{kV}/\text{Mvarh}$
    \end{tabular}


Für einen Arbeitspunkt A wird die Blindenergie, die der Überschreitung des blauen Bereichs entspricht, in Rechnung gestellt.

\end{minipage}