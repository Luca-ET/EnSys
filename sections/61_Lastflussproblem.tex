\newcolumn
\section{Lastflussproblem (P-Q Problem)}


\subsection{Problemformulierung}

\includegraphics[width=0.98\columnwidth, align=c]{images/Problemstellung_1.png}

\vspace{0.15cm}

\includegraphics[width=0.98\columnwidth, align=c]{images/Problemstellung_2.png}

\subsubsection{Iterationsformel}

Spannung $\underline{U}_2$ kann mittels Iteration approximiert werden.

\includegraphics[width=0.5\linewidth]{PQIteration.png}



\subsection{Analyse der Spannung-Leistung Verhältnisses}

$
\boxed{
P_{\text{last}} = -U_1 \cdot U_2 \cdot \frac{1}{X_L} \cdot \sin{\delta}
}
$

$
\boxed{
Q_{\text{last}} = -U_1^2 \cdot \frac{1}{X_L} + U_1 \cdot U_2 \cdot \frac{1}{X_L} \cdot \cos{\delta}
}
$

\vspace{0.15cm}

Nach der elimination von $\delta$ ergibt sich:

\vspace{0.15cm}

$
\boxed{
P_{\text{last}}^2 + \left(Q_{\text{last}} + \frac{U_2^2}{X_L} \right)^2 - \frac{U_1^2 \cdot U_2^2}{X_L^2} = 0
}
$

\vspace{0.15cm}

$
\boxed{
U_2 = \sqrt{
\frac{U_1^2}{2} - Q_{\text{last}} \cdot X_L 
\pm \sqrt{
\frac{U_1^4}{4} - P_{\text{last}}^2 \cdot X_L^2 - Q_{\text{last}} \cdot U_1^2 \cdot X_L}}}
$

\vspace{0.15cm}

Voraussetzung, dass mindestens eine Lösung existiert, ist:

\vspace{0.15cm}

$
\boxed{
\left(2 \cdot Q_{\text{last}} \cdot X_L - U_1^2 \right)^2 - 4 \cdot X_L^2 \cdot \left(P_{\text{last}}^2 + \left(Q_{\text{last}}\right)^2 \right) \geq 0
}
$

\vspace{0.15cm}
\begin{minipage}[c]{0.5\columnwidth}
    \begin{itemize}
      \item 2 Lösungen für $U_2$
      \item Stabile Lösung bei hohen Spannungen
  \end{itemize}
\end{minipage}
\begin{minipage}[c]{0.5\columnwidth}
    \begin{itemize}
        \item Instabile Lösung bei niedrigen Spannungen
        \item Die PV-Kurve wird auch „Nasenkurve” genannt
    \end{itemize}
\end{minipage}

\subsubsection{PV-Kurve, Nasenkurve}

\includegraphics[width=0.75\columnwidth, align=c]{images/Nasenkurve.png}






