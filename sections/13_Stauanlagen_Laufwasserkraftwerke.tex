\section{Laufwasserkraftwerke}


\subsection{Stauanlagen bei Laufwasserkraftwerken}


\subsubsection{Unterschied zu Speicherkraftwerken}
Das Wasser muss immer fliessen können – es kann weder gestoppt noch umgeleitet werden.

\subsubsection{Abfluss}
Muss auch während Instandhaltungsarbeiten gewährleistet sein. Eine mindestens (n-1) -Sicherheit muss für wehrabschlüsse vorhanden sein.
Auch beim Ausfall eines Ablassorgans (Verstopfung) muss das definierte Höchsthochwasser beherrschbar sein.

\subsection{Ort des Maschinenhauses}
Maschinenhaus auf der Kurvenaussenseite

\textbf{\textcolor{green}{Vorteile:}}  
\begin{itemize}
    \item Druckhöhe vor den Trubinen ist grösser
\end{itemize}

\textbf{\textcolor{red}{Nachteile:}}  

\begin{itemize}
    \item Geschwemsel lagert sich vor allem vor dem Rechen ab
    \item Bei starkem Geschwemseltrieb muss eventuell sogar das Kraftwerk abgestellt werden
\end{itemize}
\subsection{Wasserfassung}
Der Wasserhaushalt der Anlagen kann durch Ausleiten von Wasserfassungen beeinflusst werden. \\
Das ist wichtig bei:
\begin{itemize}
    \item Hochwasser, keine Konzentrationen verursachen
(Abfluss durch das natürliche Bachbett und keine Konzentration auf einen Ort, z.B. auf ein Ausgleichsbecken, welches überläuft)
\item Verhinderung von Geschiebeeintrag in das Stollensystem bei Hochwasser ($\Rightarrow$ wesentliche Minderung des Aufwands für die Wiederinbetriebnahme nach dem Hochwasser)
\end{itemize}
\begin{minipage}[c]{0.5\columnwidth}
    \subsubsection{Seitenentnahme}
    \includegraphics[width=1\linewidth]{images/Seitenentnahme.png}
\end{minipage}
\begin{minipage}[c]{0.5\columnwidth}
    \subsubsection{Stirnentnahme}
    \includegraphics[width=1\linewidth]{images/Stirnentnahme.png}
\end{minipage}

\begin{minipage}[c]{0.5\columnwidth}
\subsubsection{Sohlenentnahme}
\includegraphics[width=1\linewidth]{images/Sohlenentnahme.png}
\end{minipage}
\begin{minipage}[c]{0.5\columnwidth}
    \begin{itemize}
        \item Stäbe längs zur Strömung
        \item Für steile Gefälle und viel Geröll
        \item Kies und Sand kommen weiter durch
    \end{itemize}
\end{minipage}

\subsubsection{Coanda Rechen}
\includegraphics[width=1\linewidth]{images/CondaRechen.png}
\begin{itemize}
    \item Rechen wie ein „Mehrklingen-Rasierer“, Wasser wird abgeschert (Coanda Effekt)
    \item Fallhöhenverlust, daher nur in Mittel- und Hochdruckkraftwerken
    \item Stababstand $1\, \mathrm{mm}$
    \item $90\, \mathrm{\%}$ der Feststoffe, die grösser als $0,5\, \mathrm{mm}$ sind,
werden abgehalten $\Rightarrow$ oft wird kein Entsander mehr benötigt
\item Schluckvermögen $140\, \mathrm{l/s}$ pro Laufmeter
\item Benötigtes Gefälle von $1.3\, \mathrm{m}$
    
\end{itemize}
\subsection{Entsander}
Funktionsprinzip im Allgemeinen: Wassergeschwindigkeit so verkleinern, dass Körper einer bestimmten Korngrösse sich setzen können.
Das automatische Spülen der Wasserfassung stellt eine Gefahr für Personen im Bachbett in unmittelbarer Nähe des Auslaufes dar.
\includegraphics[width=1\linewidth]{images/Entsander.png}
Der Wasserhaushalt der Anlagen kann durch Ausleiten von Wasserfassungen beeinflusst werden. Das ist wichtig bei:
\begin{itemize}
    \item Hochwasser, keine Konzentrationen verursachen
(Abfluss durch das natürliche Bachbett und keine Konzentration auf einen Ort, z.B. auf ein Ausgleichsbecken, welches überläuft)
\item Verhinderung von Geschiebeeintrag in das Stollensystem bei Hochwasser ($\Rightarrow$ wesentliche Minderung des Aufwands für die Wiederinbetriebnahme nach dem Hochwasser)
\end{itemize}
\subsection{Verhinderung von Schwall und Sunk}
\textbf{Schwall} kann zu Überschwemmungen führen \textbf{Sunk} kann Schiffe auf Grund setzten.
\begin{itemize}
    \item Wasserstrom durch das KW nicht plötzlich stoppen, sondern zunächst Turbinenleistung beibehalten
    \item Heizen des Flusswassers mit  Wasserwiderstand (wie Tauchsieder)
    \item Danach Leistung allmählich zurückfahren
    \item Ermöglicht sprungfreie Umlagerung des Wasserstroms vom Kraftwerkskanal ins alte Flussbett
\end{itemize}
\subsection{Abfluss- /Leistungsregelung}
Funktionen Pegelregler:
\begin{itemize}
    \item Zuordnung auf die einzelnen Stellorgane (Logik-Verteilfunktion im Sinne der Prioritäten)
    \item Kontinuierliche Regelung der einzelnen Stellorgane
\end{itemize}
\includegraphics[width=0.8\linewidth]{images/AbflussLeistungsregelung.png}

\subsection{Flusskraftwerkstypen}
\textbf{Ausleitungskraftwerke (Umleitungskraftwerke)}

\includegraphics[width=0.6\columnwidth, align=c]{images/Ausleitungskraftwerke.jpg}
\begin{itemize}
  \item Bessere Ausnutzung des Gefälles in flachen Tälern
  \item Wasserwirtschaftliche Belange
  \item Aspekte hinsichtlich des Grundwassers sowie kulturtechnische Erwägungen
  \item Einfacher zum Bauen
\end{itemize}
\textbf{Blockbauweise}

\includegraphics[width=0.6\columnwidth, align=c]{images/Flusskraftwerke_Bauweise_1.jpg}\\

\textbf{Zwillingsbauweise}

\includegraphics[width=0.6\columnwidth, align=c]{images/Flusskraftwerke_Bauweise_2.jpg}\\

\textbf{Buchtenkraftwerk}

\includegraphics[width=0.6\columnwidth, align=c]{images/Flusskraftwerke_Bauweise_3.jpg}\\


Moderne Flusskraftwerke werden heute gemäss Bauweise 1 gebaut.

Der Wirkungsgrad bei Bauweise 2 und 3 ist schlechter als bei Bauweise 1.


\subsection{Arten von Wehr- und Sektorverschlüssen}
\includegraphics[width=0.98\columnwidth, align=c]{images/Arten_Wehr_und_Sektorverschlüsse_1.png}

\includegraphics[width=0.98\columnwidth]{images/Arten_Wehr_und_Sektorverschlüsse_2.png}


\newcolumn