\section{Atomkraftwerk}

\subsubsection{Merkmale Nukleare Dampferzeugung}

\begin{outline}
  \1 Leistungsfähige Energiequelle
  \1 CO2 - freie Produktion elektrischer Energie
  \1 Aufwändige Technologie
  \1 Sicherheit
  \1 Tiefenlager radioaktiver Stoffe
  \1 Diskussion in Politik, Gesellschaft, Ethik
\end{outline}


\subsubsection{Kernprozesse für die Energiegewinnung}

\begin{outline}
  \1 Künstliche Kernspaltung schwerer Kerne (Fission)
    \2 → Kernkraftwerke 3. Generation
    \2 (Stand der Technik)
    
  \1 Umwandlung von schweren Kernen in gut spaltbare Kerne im Brutprozess (Konversion)
    \2 → Kernkraftwerke 4. Generation
    \2 (in Entwicklung)
    
  \1 Verschmelzung leichter Kerne zu einem Kern (Fusion)
    \2 → Grundlagenforschung in Bearbeitung
\end{outline}


\subsection{Kernphysikalische Grundlagen}

$\boxed{A = Z + N}$ \quad Nuklid-Schreibweise: $\boxed{\text{\ce{^{A}_{Z}Element}} }$ \quad \text{z.\,B.} \quad \text{\ce{^{235}_{92}U}}

\vspace{1em}
\renewcommand{\arraystretch}{1.2}
\begin{tabular}{@{} l p{6cm} l @{}}
    $[A]$ & Anzahl Kerneteilchen eines Atoms \dotfill & $-$ \\
    $[Z]$ & Anzahl Protonen (Kernladungszahl) \dotfill & $-$ \\
    $[N]$ & Anzahl Neutronen \dotfill & $-$ \\
\end{tabular}



\subsection{Spaltung schwerer Kerne}

\begin{outline}
    \1 Spaltung schwerer Kerne
    \1 Einige Isotope besitzen die Eigenschaft, dass sie beim Beschießen mit langsamen Neutronen diese im Kern absorbieren und in zwei Tochterkerne zerfallen, wobei gleichzeitig 2–3 Neutronen frei werden.

    \vspace{0.2cm}

    $\boxed{
    \ce{^{235}_{92}U + ^{1}_{0}n \rightarrow  ^{89}_{36}Kr + ^{144}_{56}Ba + 3 ^{1}_{0}n + 200MeV}
    }$

    \vspace{0.2cm}

    \1 Bindungsenergie wird dabei frei.\\
        Im Mittel sind dies: 
        \textbf{200 MeV = $\bm{3{,}2 \cdot 10^{-11}}$ Ws pro Spaltung}
    \1 Die „schnellen“ Neutronen müssen abgebremst werden ($\Rightarrow$  thermische Neutronen), so dass der Prozess nicht abbricht.\\
    Dies geschieht mit einem Moderator wie „leichtes“ Wasser oder Graphit.
    \1 Werden genügend thermische Neutronen zur Verfügung gestellt, hält sich durch eine Kettenreaktion der Spaltungsprozess selbst aufrecht.
\end{outline}


\crd{To be continued ...}




















