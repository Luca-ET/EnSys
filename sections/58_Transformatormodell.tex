\section{Transformator}

\subsection{Idealer Transformator}

\begin{minipage}[t]{0.25\columnwidth}
    \includegraphics[width=1\columnwidth, align=c]{images/Idealer_Transformator_1.png}
\end{minipage}
\hfill
\begin{minipage}[t]{0.22\columnwidth}
    $
        \boxed{\frac{\underline{U_1}}{\underline{U_2}} = \frac{\underline{I_2}}{\underline{I_1}} = \frac{N_1}{N_2} =t}
    $
\end{minipage}
\hfill
\begin{minipage}[t]{0.23\columnwidth}
    $\boxed{U_n = \omega \cdot N_n \cdot \frac{\phi}{\sqrt{2}}}$
\end{minipage}
\hfill
\begin{minipage}[t]{0.25\columnwidth}
    $\boxed{\phi = B \cdot A_{FE}}$\\
    $\boxed{A_{FE}=\frac{\pi D\cdot k_a \cdot f_{FE}}{4}}$
\end{minipage}
\hfill
\begin{tabular}{@{} l p{6cm} l @{}}
    $[\underline{U_1}]$   & Primärspannung \dotfill & $\mathrm{V}$ \\
    $[\underline{I_1}]$   & Primärstrom \dotfill & $\mathrm{A}$ \\
    $[N_1]$   & Anz. Primärwindungen \dotfill & $-$ \\
    $[\underline{U_2}]$     & Sekundärspannung \dotfill & $\mathrm{V}$\\
    $[\underline{I_2}]$   & Sekundärstrom \dotfill & $\mathrm{A}$\\
    $[N_2]$   & Anz. Sekundärwidungen \dotfill & $n$ \\
    $[N_n]$   & Anz. Windungen an Ausgang n \dotfill & $n$ \\
    $[\phi]$   & Magnetischer Fluss \dotfill & $\mathrm{Wb}$ \\
    $[B]$   & Magnetische Flussdichte \dotfill & $\mathrm{T}$ \\
    $[A_{FE}]$  & Wirksamer Eisenquerschnitt \dotfill & $\mathrm{m^2}$ \\
    $[D]$  & Durchmesser \dotfill & $\mathrm{m}$ \\
    $[k_{a}]$  & Kernquerschnitt \dotfill & $-$ \\
    $[f_{FE}]$  & Eisfüllfaktor \dotfill & $-$ \\
\end{tabular}
\subsection{Verluste im Transformator}

\includegraphics[width=0.98\columnwidth, align=c]{images/TrafoBild.png}

\vspace{0.15cm}
\begin{minipage}[t]{0.4\columnwidth}
    \begin{itemize}
        \item Streuverluste in $L_{\sigma i}$
        \item Wicklungsverluste in $R_{i}$
        \item Kern-/Hystereseverluste in $R_{FE}$
    \end{itemize}
\end{minipage}
\begin{minipage}[t]{0.2\columnwidth}
    \begin{minipage}[t]{1\columnwidth}
        $\boxed{L_{\sigma2}' = u^2 L_{\sigma2}}$
    \end{minipage}
    \hfill
    \begin{minipage}[t]{1\columnwidth}
        $\boxed{R_{2}' = u^2 R_{2}}$
    \end{minipage}
\end{minipage}
\begin{minipage}[t]{0.2\columnwidth}
    \begin{minipage}[t]{1\columnwidth}
        $\boxed{I_2' = uI_2}$
    \end{minipage}
    \begin{minipage}[t]{1\columnwidth}
        $\boxed{U_2' = uU_2}$
    \end{minipage}
\end{minipage}
\begin{minipage}[t]{0.2\columnwidth}
    $\boxed{\mathrm{u=t}}$
\end{minipage}
\begin{itemize}
    \item Für $\mathbf{L_h}$ und $\mathbf{R_{Fe}}$ $\Rightarrow$ Leerlauf
    \item Für $\mathbf{R_1}$, $\mathbf{L_{\sigma1}}$, $\mathbf{R_2'}$, $\mathbf{L_{\sigma_2}'}$ $\Rightarrow$ Kurzschluss
\end{itemize}
\subsection{Praktisches Transformatormodell}

\includegraphics[width=0.98\columnwidth, align=c]{images/Praktisches_Transformatorbild_2.png}

\vspace{0.15cm}

\begin{itemize}
    \item \( Z_h \gg Z_t \Rightarrow Z_h \) vernachlässigen
    \item \( I_h \approx \,\%\!1 \) von \( I_t \) bei großen Transformatoren
    \item Kernverluste
\end{itemize}


\subsection{Umrechnung von Impedanzen}


\begin{minipage}[t]{0.48\columnwidth}
    \includegraphics[width=0.98\columnwidth, align=c]{images/Umrechnung_Impedanzen_1.png}
\end{minipage}
\hfill
\begin{minipage}[t]{0.48\columnwidth}
    \includegraphics[width=0.98\columnwidth, align=c]{images/Umrechnung_Impedanzen_2.png}
\end{minipage}

\vspace{0.15cm}

$
    \boxed{\frac{\underline{Z_1}}{\underline{Z_2}} = t^2}
$

\subsection{Leerlauf / Kurzschluss}

\textbf{Leerlauf}\\
Nur $L_h$ und $R_{FE}$ müssen beachtet werden.\\
$\boxed{\underline{I}_{1} = \underline{I}_{R_{FE}}+\underline{I}_{L_{h}}}$ $\Rightarrow$ \boxed{|\underline{I}_{1}| = \sqrt{|\underline{I}_{R_{Fe}}|^2+|\underline{I}_{L_{h}}}|^2}\\
\textbf{Kurzschluss}\\
$L_h$ und $R_{FE}$ nicht beachten $\Rightarrow$ haben praktisch keinen Einfluss.\\
Kurzschlussspannung $U_k = x\%$ von Nennspannung $\underline{U}_1$
so dass bei Kurzschluss Nennstrom fliesst, $P_{nenn(R_1+R_2)}=I_{1n}\cdot U_{1n(R_1+R_2)}=P_{ks(R_1+R_2)}$ Kupferverl. bleiben gleich.

\subsection{Dreiphasentransformatoren}

\begin{itemize}
    \item Verschaltung der drei Phasenwicklungen auf Primär- und Sekundärseite wirkt sich auf Übersetzungsverhältnis aus.
    \item Parallelschaltung nur möglich, wenn sekundärseitig gleiche „Zahl“ (Bsp. Yy0 und Dd0)
    \item Amplitude und Phasenlage der Spannung können verändert werden.
    \item Übersetzungsverhältnis wird komplex: \( t \)
\end{itemize}

\vspace{0.15cm}
\subsection{Schaltgruppen}

\begin{minipage}[c]{0.48\columnwidth}
    \myul{\textbf{Mögliche Schaltungen}}\\
\end{minipage}
\hfill
\begin{minipage}[c]{0.48\columnwidth}
    \myul{\textbf{Bezeichnung}}\\
\end{minipage}

\begin{minipage}[c]{0.48\columnwidth}
    \begin{itemize}
        \item Y \dots\ Sternschaltung
        \item D \dots\ Dreieck-Schaltung
        \item Z \dots\ „Zick-zack“ (Nur US)
    \end{itemize}
\end{minipage}
\hfill
\begin{minipage}[c]{0.48\columnwidth}
    \begin{itemize}
        \item 1. Buchstabe (groß): \\
        Schaltung Oberspannungsseite (OS)
        \item 2. Buchstabe (klein): \\
        Schaltung Unterspannungsseite (US)
        \item Zahl (n) : Phasendrehung/Phasenverschiebung gegenüber OS: $n\cdot\angle30$
    \end{itemize}
\end{minipage}

\includegraphics[width=0.98\columnwidth, align=c]{images/Ansteuerung.png}\\

\begin{minipage}[c]{1\columnwidth}
    \myul{\textbf{Übersetzungsverhältnisse:}}\\
    \begin{tabular}{c l c}
        Schaltgruppe & Verhälnis\\
        \hline
        Yy0 & $U_{1n} = t\cdot U_{2n}$ &\\
         & $U_{1} = t\cdot U_{2}$ &\\
        \hline
        Dy5 & $U_{1n} = \frac{t\cdot U_{2n}}{\sqrt{3}}$&$N_{1n}=\sqrt{3}\frac{U_{1n}\sqrt{2}}{\omega\,\phi} $\\
        \hline
        Yd5 & $U_{1n} = t\cdot U_{2n}\cdot \sqrt{3}$&$N_{1n}=\frac{U_{1n}\sqrt{2}}{\omega\,\phi\sqrt{3}} $\\
        \hline
        Yz5 & $U_{1n} = \frac{\sqrt{3}\cdot}{2}U_{2n}$&\\
    \end{tabular}
\end{minipage}













