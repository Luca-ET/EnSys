\section{Schutztechnik}


\subsection{Aufgaben der Schutztechnik}

\begin{itemize}
  \item Gewährleistung der Kontinuität des Betriebs und die Stabilität des Netzes
  \item \textbf{Sicheres, schnelles} und \textbf{selektives} Abschalten der gestörten Netzelemente
  \item Schutzsystem muss die Fähigkeit haben, den \textbf{Fehlerzustand} vom \textbf{normalen Betriebszustand} genau zu unterscheiden.
  \item Einleitung von \textbf{Massnahmen}, die sich möglichst eng begrenzt \textbf{auf den Fehlerort} beziehen
\end{itemize}



\subsection{Selektivität}

\begin{itemize}
  \item Selektivität gewährleistet, dass im Fehlerfall möglichst nur der fehlerhafte Netzabschnitt oder Verbraucher abgeschaltet wird und andere Netzabschnitte und Verbraucher nicht mit beeinträchtigt werden.
  \item Erreichen durch \textbf{Staffelung}
  \begin{itemize}
    \item Zeitstaffelung
    \item Stromstaffelung
    \item Impedanzstaffelung
  \end{itemize}
  \item zeitverkürzte Selektivität erzielen:
  Die zeitverkürzte Selektivitäts-Steuerung (ZSS) nutzt eine Kommunikation zwischen den Leistungsschaltern, um unnötige Verzögerungen zu vermeiden.
\end{itemize}



\subsection{Überstrom-Zeit-Schutz}
Die Auswertung der Stromamplitude ist die einfachste Methode den normalen Betriebszustand eines Netzes von einem Netzfehler zu unterscheiden.

\vspace{0.15cm}

\textbf{Anwendung:} Leitungs-, Generator-, Transformator- und Sammelschienenschutz

\subsection{Unabhängiger Maximalstrom Zeitschutz - UMZ}
Abschnittweise definiert\\

\includegraphics[width=0.98\columnwidth, align=c]{images/Schutztechnik_1.png}

\vspace{0.15cm}
\subsection{Abhängiges Maximalstrom Zeitschutz - AMZ}
Als Funktion definiert\\
$
\boxed{
t(I) = \frac{0.14}{\left( \frac{I}{I_P} \right)^{0.02} - 1} \cdot T_P
}
$

\vspace{0.15cm}

\renewcommand{\arraystretch}{1.2}
\begin{tabular}{@{} l p{6.5cm} l @{}}
    $[t(I)]$      & Auslösezeit \dotfill                      & $\text{s}$ \\
    $[I]$         & Strom \dotfill                            & $\text{A}$ \\
    $[I_P]$       & Parametrierbarer Anregestrom \dotfill     & $\text{A}$ \\
    $[T_P]$       & Multiplikator \dotfill                    & $\text{s}$ \\
\end{tabular}



\subsection{Zeitstaffelung}

\includegraphics[width=0.8\columnwidth, align=c]{images/Schutztechnik_3.jpg}

\vspace{0.15cm}

\begin{minipage}[c]{0.48\columnwidth}
    %\begin{center}
        \includegraphics[width=0.98\columnwidth, align=c]{images/Schutztechnik_4.jpg}
    %\end{center}    
\end{minipage}
\hfill
\begin{minipage}[c]{0.38\columnwidth}
    %\begin{center}
        \includegraphics[width=0.98\columnwidth, align=c]{images/Schutztechnik_5.jpg}
    %\end{center} 
\end{minipage}

\vspace{0.15cm}

\textbf{\textcolor{green}{Vorteile:}} 
\begin{itemize}
    \item \textbf{Einfache Methode}, den normalen Betriebszustand eines Netzes von einem Netzfehler zu unterscheiden.
\end{itemize}

\vspace{0.15cm}

\textbf{\textcolor{red}{Nachteile:}} 
\begin{itemize}
    \item Wenn die \textbf{Einspeisung} sich \textbf{ändert}, würde der Strang unter Umständen \textit{unselektiv} abgeschaltet.
    \item In diesem Fall ist eine \textbf{Fehlerortung nicht mehr möglich}, da nur die Auslösezeit als Hilfsgrösse zur Selektion des Fehlerorts dient.
    \item Die Auslösezeit nahe der Einspeisung wird immer grösser und somit fliessen \textbf{grössere Kurzschlussströme} eine \textbf{verhältnismässig lange Zeit}.
    \item Nur in \textbf{radialen Systemen} (Strahlennetz) verwendet.
\end{itemize}


\subsection{Distanzschutz}

\includegraphics[width=0.98\columnwidth, align=c]{images/Distanzschutz_1.jpg}

\begin{itemize}
    \item \textbf{Nachteile des Überstromzeitschutzes}, wie etwa lange Auslösezeiten oder fehlende Selektivität bei Abweichung vom Normalschaltzustand, ist \textbf{behoben}.
    \item \textbf{Zeitstaffelschutz}, dessen Auslösezeit mit grösser werdender Entfernung zwischen Fehlerstelle und Schutzgerät-Einbauort stufig ansteigt.
    \item \textbf{Distanz \( \equiv \) Impedanz}
    \item \textbf{Reiner Kurzschlussschutz.}
    \item Aus den Messwerten der \textbf{Strom-} und \textbf{Spannungswandler} wird die \textbf{Impedanz} ermittelt.
    \item $Z_1$: Fehler nah, $Z_3$ Fehler weit weg
    \item Fehler vorwärts oder rückwärts ist aufgrund der Stromrichtung 
\end{itemize}


\subsubsection{Staffelplan des Distanzschutzes}
\includegraphics[width=0.98\columnwidth, align=c]{images/Distanzschutz_2.jpg}

\begin{itemize}
    \item selektive Arbeitsweise des Distanzschutzes
    \item Auslösezeiten \( t_A \) in Abhängigkeit von Entfernung des Fehlers vom Relaisort
    \item Die stufenförmige Charakteristik der Auslösekennlinien sorgt dafür, dass Netzfehler schneller abgeschaltet werden
    \item Die vorgelagerten Schutzgeräte dienen als Reserveschutz
    \item Fehlerortung viel Präziser
\end{itemize}




\subsection{Differentialschutz}

\textbf{Prinzip:}
\begin{itemize}
    \item Das Erkennungsprinzip beruht auf dem ersten kirchhoffschen Gesetz. 
    \textit{„Die Summe der zufließenden Ströme in einem elektrischen Knotenpunkt muss gleich der Summe der abfließenden Ströme sein”}
    \item \( I_{\text{Diff}} = |I_1 + I_2| > 0 \) ist Fehlerfall
    \item Wird bei Transformatoren angewendet
\end{itemize}
\includegraphics[width=0.45\columnwidth, align=c]{images/Differentialschutz.jpg}
\textbf{Vorteile:}
\begin{itemize}
    \item eine unverzögerte Abschaltung an jeder beliebigen Stelle des Schutzbereiches
    \item Die Abgrenzung des Schutzbereichs an den Leitungsenden. Damit sind, im Gegensatz zum Distanzschutz, 100\,\% der Leitung in Schnellzeit geschützt.
\end{itemize}

\textbf{Nachteile:}
\begin{itemize}
    \item keine Reserveschutzfunktionen für die angrenzenden Bereiche.
\end{itemize}

\subsection[Schwefelhexafluorid (SF6)]{Schwefelhexafluorid ($\mathrm{SF_6}$)}

+ Sehr gute elektrische Isolierfähigkeit (hohe Durchschlagfestigkeit). \\
+ Hervorragende Lichtbogenlöscheigenschaften.\\
+ Nicht brennbar, ungiftig (im reinen Zustand). \\
- Sehr hohes Treibhauspotenzial (rund 23.500x stärker als CO2). \\
- lange Lebensdauer (ca. 3.200 Jahre).\\
- Bildung giftiger Zersetzungsprodukte bei Lichtbogenentladung.\\
