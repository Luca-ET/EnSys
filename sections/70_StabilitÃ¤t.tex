\section{Stabilitätsbegriff}

Die Stabilität des Stromnetzes ist die Fähigkeit eines elektrischen Stromnetzes, bei einem gegebenen Ausgangszustand nach einer physikalischen Störung wieder einen Gleichgewichtszustand zu erreichen, wobei die meisten Systemvariablen so begrenzt sind, dass praktisch das gesamte System intakt bleibt.

\subsection{Aufgaben des Netzbetriebs}

\begin{tabular}{|l l|}
\hline
\textbf{{Technische Ziele:}} & \textbf{{Zu jedem Zeitpunkt Sicherstellung von ...}} \\
\hline
\textbf{Leistungsbilanz:} & Angebot = Nachfrage \\
\textbf{Synchronizität und Dämpfung:} & gleiche Frequenz an allen Knoten \\
\textbf{Frequenzstabilität:} & Frequenz = 50 Hz \\
\textbf{Spannungsstabilität:} & Spannung = Nennspannung \\
\hline
\end{tabular}

\vspace{0.15cm}

\begin{tabular}{|l l|}
\hline
\textbf{Technische Herausforderungen:} & \textbf{Beschreibung} \\
\hline
\textbf{Variabilität} & Fluktuierende Last- und Angebotsverläufe \\
\textbf{Unsicherheit} & Ungenaue Last- und Angebotsvorhersagen \\
\textbf{Störungen} & Ausfälle, Leistungsoszillationen, Blackouts \\
\hline
\end{tabular}



\subsection{}


\subsection{}


\subsection{}


\subsection{}


\subsection{}


\subsection{}


\subsection{}


\subsection{}


\subsection{}


\subsection{}
