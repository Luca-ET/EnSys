\section{Thermische Kraftwerke/ Dampfkraftwerke}
Gegen 90\% der weltweiten Bereitstellung
elektrischer Energie erfolgt in «Thermischen Kraftwerken».\\
Kraftwerksarten:
\begin{itemize}
    \item Thermische Kraftwerke auf fossiler Basis
(Dampf-, Gasturbinen-, Gas- und Dampfturbinen-Kraftwerke (GUD), Blockheiz-, Diesel- Kraftwerke)
    \item Kernkraftwerke
    \item Geothermische Anlagen
    \item Solarthermische Anlagen
    \item Heizkraftwerke
    \item Kehrichtverbrennung mit thermischem Kraftwerk
\end{itemize}

\subsection{Funktionsweise}
\includegraphics[width=0.95\columnwidth, align=c]{images/Prinzip_Thermisch.png}\\
\includegraphics[width=0.5\columnwidth, align=c]{images/Prinzip2_Thermisch.png}\\
\begin{enumerate}
    \item Brenn- und Treibstoffe, Geothermie, Solarthermie, atomare Bindungsenergie
    \item Thermische Energie in Form von Gas oder Dampf
    \item Gas- oder Dampfturbine
    \item Generator
    \item Transformator/Netz
\end{enumerate}

\subsection{Thermodynamik}
\vspace{-0.25cm}

    \includegraphics[width=0.98\linewidth, align=c]{images/Thermo.png}\\ 
    \textbf{isotherme:} $T = const,\ W = m \cdot Ri \cdot T \cdot \mathrm{ln}\left(\frac{p2}{p1} \right)$\\
    \textbf{isochore:} $V = const,\ Q = c_v \cdot m \cdot \Delta T$\\
    \textbf{isobare:} $p = const,\ Q = c_p \cdot m \cdot \Delta T$\\
    \textbf{isotrop:} Entropie bleibt konstant $ \frac{T1}{T2} = \left( \frac{p_1}{p_2}\right) ^{\frac{x-1}{x}}= \frac{T_4}{T_3}$\\
    \textbf{adiabatisch:} Zustandsänderung ohne Wärmetausch mit der Umgebung\\
    \textbf{ideales Gas:} $\frac{p\cdot v}{T} = const$\\
    \textbf{Enthalpie:} Wärmeinhalt, $H=U+p\cdot V$, $U$: innere Energie\\
    \textbf{Entropie:} Energie pro Temperatur $T$, Entropieänderung $\Delta S =\frac{Q}{T}$, $Q$: zugeführte Wärme
\subsection{Kreisprozesse}
\subsubsection{Carnot Prozess}
Wärme (Dampferzeuger) -> Mechanische Arbeit (Dampfturbine)\\
\begin{minipage}[ht]{0.49\columnwidth}
    \includegraphics[width=0.98\linewidth, align=c]{images/carnot1.png}
    
\end{minipage}
\begin{minipage}[ht]{0.49\columnwidth}
    \textbf{Carnot-Prozess im T, s - Diagramm}\\
    Nutzbare Arbeit: $w_nutz = q_{zu} - q_{ab}$\\Da die Entropie als $dq = T \cdot ds$ definiert ist, erscheint sowohl die dem Prozess zugeführte Wärme $q_{zu} = T_{max} ( s_3 - s_2 )$ als auch die abgeführte Wärme $q_{ab} = T_{min} (s_4 - s_1)$  als Fläche im T, s -Diagramm.\\
    Der Wirkungsgrad des Carnot-Prozesses ergibt sich somit:
\end{minipage}

\textbf{Thermischer Wirkungsgrad}\\
$\boxed{\eta_{\mathrm{th}}' 
= \frac{P_m - P_u}{P_{zu}}
= \frac{(h_3 - h_{4^*})\,\dot m_D - (h_{2^*} - h_1)\,\dot m_D}{(h_3 - h_{2^*})\,\dot m_D}}$

$\boxed{\eta_{\mathrm{th}}'
= 1 - \frac{P_{ab}}{P_{zu}}
= 1 - \frac{(h_{4^*} - h_1)\,\dot m_D}{(h_3 - h_{2^*})\,\dot m_D}
= 1 - \frac{h_{4^*} - h_1}{h_3 - h_{2^*}}}$

\begin{tabular}{@{} l p{7.5cm} l @{}}
  $[P_m]$ & abgegebene mechanische Leistung der Wärmekraftmaschine \dotfill & $\mathrm{kW}$    \\
  $[P_{u}]$ & Leistung der (elektrisch angetriebenen) Speisewasserpumpe \dotfill & $\mathrm{kW}$    \\
  $[P_{zu}]$ & Leistungszufuhr aus dem Kessel \dotfill & $\mathrm{kW}$    \\
  $[P_{ab}]$ & abzuführende Wärmeleistung im Kondensator \dotfill & $\mathrm{kW}$    \\
  $[\dot m_{D}]$ & konstanter Dampf-Massenstrom pro Zeiteinheit \dotfill & $\mathrm{kg/s}$  \\
  $[h]$ & spezifische Enthalpie \dotfill & $\mathrm{kJ/kg}$ \\
  $[Q]$ & Wärmemenge \dotfill & $\mathrm{J}$ \\
\end{tabular}

\subsubsection{Wirkungsgrad erhöhen}

\begin{itemize}  
    \item Zwischenüberhitzung
    \item Speisewasservorwärmung
    \item Erhöhung des Dampfdrucks und der Temperatur
    \item Kombination mit Gasturbine (GuD-Prozess)
\end{itemize}

\subsection{Joule-Prozess für Gasturbinen}

\begin{minipage}[ht]{0.49\columnwidth}
    \includegraphics[width=0.98\linewidth, align=c]{images/carnot2.png}
    
\end{minipage}
\begin{minipage}[ht]{0.49\columnwidth}
    \textbf{Joule-Prozess im T, s - Diagramm}\\
    \textbf{Ablauf:}\\
    1 - 2 Verdichtung, $s$ const, $p$ steigt (isotrop)\\
    2 - 3 Erwärmung, $s$ steigt, $T$ steigt, $p$ konstant (isobar)\\
    3 - 4 Entspannung, $s$ konstant, $p$ sinkt (isotrop)\\
    4 - 1 Abkühlung, $s$ sinkt, $T$ sinkt, $p$ konstant (isobar)\\
\end{minipage}

\subsection{Enthalpie}
\begin{minipage}[ht]{0.49\columnwidth}
    \includegraphics[width=0.95\columnwidth, align=c]{images/Wasserverlauf.png}
    
\end{minipage}
\begin{minipage}[ht]{0.49\columnwidth}
    $\boxed{dH= dQ + V\cdot dp}$ d.h. Enthalpie ändert mit Druck\\
    Verdampfungswärme:
    $\boxed{q_V=T_V\cdot \Delta S}$\\
    \textbf{Arregatsänderungen:}\\
    Fest - Flüssig: Schmelzen/ erstarren\\
    Flüssig - Gas: Verdampfen/ kondensieren\\
    Fest - Gas: Sublimieren/ Resublimieren\\
    $0^\circ$C = 273K\\
\end{minipage}

\subsection{Mollier-Diagramm}
\includegraphics[width=0.95\columnwidth, align=c]{images/hsDiagramm.png}\\
Dampfturbine mit Zwischenüberhitzung\\
\circled{1}: Startdruck: $500\, \mathrm{^\circ C}$ $50\, \mathrm{bar}$\\
\circled{1} zu \circled{2}: Entspannung in Turbine\\
\circled{2} zu \circled{3}: Temperaturzufuhr mit konstantem Druck (isobar); Dampferzeuger(Energieeintrag,Q)\\
\circled{3} zu \circled{4}: Entspannung in Turbine\\
$\boxed{x=\frac{m_D}{m_W + m_D}}$
$\boxed{\frac{m_w}{m_d}= \frac{1}{x}-1}$
$\boxed{\Delta h_{liquid}= V \cdot \Delta p}$
($1  Bar = 10^5pa$)\\
\boxed{P = \dot{m}_D \cdot \Delta h}
$\boxed{P_{\text{mech}} = \dot{m}_k \cdot \eta_{\text{Th}} \cdot \eta_{\text{DE}} \cdot H
}$
$\boxed{\eta_{\mathrm{th}}
= 1 - \frac{P_{ab}}{P_{zu}}
= 1 - \frac{\Sigma h_{tief}}{{\Sigma h_{hoch}}}} $
$\boxed{\eta_\mathrm{c} = 1-\frac{T_U}{T_O}}$ $0^\circ C= 273 K$ \\

\textcolor{red}{Senkrechte $[\Delta h]$ nicht beachten bei der Berechnung von Wirkungsgrad des Kreisprozesses.}


\begin{tabular}{@{} l p{7.5cm} l @{}}
  $[x]$ & Wasserdampfgehalt \dotfill &    \\
  $[m_D]$ & Dampfmassegehalt \dotfill & $\mathrm{kg/m^3}$    \\
  $[m_W]$ & Wassermassegehalt \dotfill & $\mathrm{kg/m^3}$    \\
  $[\Delta h]$ & Entropie zweier Punkte senkrecht verbunden \dotfill & $\mathrm{kJ/\ kg}$   \\
  $[V]$ & Volumen $0.001\,\mathrm{\frac{m^3}{kg}}$\dotfill & $\mathrm{m^3}$ \\
  $[\Delta p]$ & Druckunterschied \dotfill & Pa \\
  $[\eta_{\mathrm{th}}]$ & Thermischer Wirkungsgrad \dotfill &  \\
  $[\eta_{\mathrm{c}}]$ & Carnot-Faktor \dotfill &  \\
  $[T_o]$ & Höchste Temperatur Prozess \dotfill & $\mathrm{K}$ \\
  $[T_u]$ & Tiefste Temperatur über x=1 Grenze \dotfill & $\mathrm{K}$ \\
  $[P_{\text{mech}}]$ & Mechanisch abgegebene Leistung \dotfill & $\mathrm{W}$ \\
  $[\dot{m}_k]$ & Massenstrom des eingesetzten Brennstoffs \dotfill & $\mathrm{kg/s}$ \\
  $[\eta_{\text{th}}]$ & Thermischer Wirkungsgrad \dotfill &  \\
  $[\eta_{\text{DE}}]$ & Wirkungsgrad der Dampferzeugung bzw. Energieumwandlung \dotfill &  \\
  $[H]$ & Heizwert des Brennstoffs \dotfill & $\mathrm{J/kg}$ \\
\end{tabular}


\subsection{Anleitung Mollier Diagramm (s - h)}
\subsubsection{Clausius-Rankine-Prozess}
Sattdampfprozess (Dampfkreislauf)\\
1 - 2 Druckerhöhung mit Speiseerhöhung (von 1 Bar bis Zieldruck, senkrecht links unten)\\
\textcolor{ForestGreen}{2 - 3 Dampferzeuger, Energiezugabe (Druck bleibt konstant)}\\
\textcolor{purple}{3 - 4 Dampfturbine, Energieumwandlung zu mechanisch (Druckabnahme, s bleibt konstant (senkrechte Linie)}\\
\textcolor{red}{4 - 1 Kondensator, zurück zum Start, Verlustenergie (Druck konstant)}\\

\includegraphics[width=0.95\columnwidth, align=c]{images/Sattdampf.pdf}\\
\includegraphics[width=0.95\linewidth]{images/DampfkraftMitZwischenueberhitzung.png}
Wirkungsgrad:\\
\begin{minipage}[c]{0.4\columnwidth}
    $\boxed{\eta_{\mathrm{th}} = 1-\frac{h_{\mathrm{4b}}-h_{\mathrm{1}}}{(h_{\mathrm{3a}}-h_{\mathrm{2}})+(h_{\mathrm{3b}}-h_{\mathrm{4a}})}}$
\end{minipage}
\hfill
\begin{minipage}[c]{0.6\columnwidth}
\includegraphics[width=1\linewidth]{images/Zwischenueberhitzung.png}
\end{minipage}
\subsection{Varianten}
\begin{itemize}
    \item Dampfkraftwerk mit Zwischenüberhitzung
    \item Dampfkraftwerke mit Abwärmenutzung
\end{itemize}

\subsection{Komponenten}
Komponenten fossil befeuerten Dampfkraftwerke
\begin{itemize}
    \item Kessel und Dampferzeuger
    \item Rauchgasreinigung
    \item Feuerungen und Brenner
    \item Dampferzeuger mit Wirbelschichtfeuerung
    \item Dampfturbinen
    \item Kühlsysteme für Dampfkraftwerke
\end{itemize}
\textbf{Kühlsysteme:}\\
\begin{itemize}
    \item Frischwasserkühlung
    \item Ablaufkühlung
    \item Nasse Rückkühlung
    \item Trockene Rückkühlung
    \begin{itemize}
        \item direktes System
        \item indirekt mit Mischkondensator
        \item indirekt mit Oberflächenkondensator
    \end{itemize}
\end{itemize}
\vspace{0.2cm}
\begin{minipage}[ht]{0.49\columnwidth}
    \textbf{Kohlestaubbrenner:}\\
\includegraphics[width=0.95\columnwidth, align=c]{images/Kohlestaub.png}\\
\end{minipage}
\begin{minipage}[ht]{0.49\columnwidth}
    \textbf{Gasreinigung:}\\
\includegraphics[width=0.95\columnwidth, align=c]{images/GAsreinigung.png}\\
\end{minipage}

\includegraphics[width=0.95\columnwidth, align=c]{images/Entschwefelung.png}\\
\textbf{Turbine:}\\
\begin{minipage}[ht]{0.49\columnwidth}
    \includegraphics[width=0.95\columnwidth, align=c]{images/dampfturbine.png}
\end{minipage}
\begin{minipage}[ht]{0.49\columnwidth}
    \begin{tabular}{ll}
        a & Gehäuse \\ 
        b & Läufer \\ 
        c & Schaufeln \\ 
        d & Einströmstützen \\ 
        e & Ausstromstutzen \\ 
        f & Zylinderschnitt \\
        g & Leitgitter\\
        h & Lauftgitter\\
        m & Massestrom medium\\
        $F_\tau$ & Tangetialkraft Laufrad \\ 
    \end{tabular}
\end{minipage}

\newcolumn