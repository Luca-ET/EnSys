\section{Frequenzregelung}
Frequenzstabilität bezieht sich auf die Fähigkeit eines Stromversorgungssystems, eine konstante Frequenz nach einem schweren Systemstörfall aufrechtzuerhalten, der zu einem signifikanten Ungleichgewicht zwischen Erzeugung und Last führt. Sie hängt von der Fähigkeit ab, das Gleichgewicht zwischen Systemgeneration und Last bei minimalem unbeabsichtigtem Lastverlust aufrechtzuerhalten bzw. wiederherzustellen.

\subsection{Modell eines einzelnen Synchrongenerators}

$
\boxed{
\frac{d\omega}{dt} = \frac{(P_{\text{Gen}} - P_{\text{Last}})}{2 \cdot H} 
}
$

\vspace{0.15cm}

\renewcommand{\arraystretch}{1.2}
\begin{tabular}{@{} l p{7cm} l @{}}
    $[\omega]$            & Elektrische Winkelgeschwindigkeit \dotfill       & $\frac{\text{rad}}{\text{s}}$ \\
    $[t]$                 & Zeit \dotfill                                    & $\text{s}$ \\
    $[H]$                 & Trägheitskonstante (proportional zur Schwungmasse) \dotfill & $\text{s}$ \\
    $[P_{\text{Gen}}]$    & Erzeugte Leistung von der Turbine \dotfill       & $\text{W}$ \\
    $[P_{\text{Last}}]$   & Elektrische Lastleistung (Netzeinspeisung) \dotfill & $\text{W}$ \\
\end{tabular}



\subsection{}


\subsection{}


\subsection{}


\subsection{}


\subsection{}


\subsection{}


\subsection{}


\subsection{}


\subsection{}
