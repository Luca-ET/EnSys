\section{Windenergie}


\subsection{Windeleistung}
\begin{minipage}[ht]{0.59\columnwidth}
    \includegraphics[trim={0 0 3cm 0},clip ,width=0.95\columnwidth, align=c]{images/WEK_Leistungs-und_Drehzahlregelung.png}
\end{minipage}
\begin{minipage}[ht]{0.39\columnwidth}
    $
\boxed{
P_{\text{max}} = \frac{dW}{dt} = \frac{A \cdot \rho_{\text{Luft}}}{2} \cdot v_1^3
}$\\
$\boxed{
P_W = \frac{P_{\mathrm{el}}}{\eta}=c_P \cdot \frac{A \cdot \rho_{\text{Luft}}}{2} \cdot v_1^3}$\\
\textcolor{red}{Achtung! $v_1$ ist hoch 3!}
$\boxed{
W = \eta \cdot P_W \cdot \Delta t
}$\\
$\boxed{
T = \frac{W_{\text{el pro Jahr}}}{P_{\text{max}}}
}$\\
\end{minipage}







\renewcommand{\arraystretch}{1.2}
\begin{tabular}{@{} l p{8cm} l @{}}
    $[P_{\text{max}}]$ & Theoretische Windleistung \dotfill & $W$ \\
    $[P_W]$            & Effektiv nutzbare praktische Windleistung \dotfill & $W$ \\
    $[c_P]$            & Leistungsbeiwert, $c_P = 0.4 \dots 0.5$ \dotfill & $-$ \\
    $[A]$              & Rotorfläche (projizierte Fläche senkrecht zur Strömung) \dotfill & $\mathrm{m^2}$ \\
    $[\rho_{\text{Luft}}]$ & Dichte Luft, $\approx 1{,}29 \, \frac{\mathrm{kg}}{\mathrm{m^3}}$ \dotfill & $\frac{\mathrm{kg}}{\mathrm{m^3}}$ \\
    $[v_1]$            & Anströmgeschwindigkeit des Windes \dotfill & $\frac{\mathrm{m}}{\mathrm{s}}$ \\
    $[\eta]$           & Gesamtwirkungsgrad der Umwandlungskette \dotfill & $-$ \\
    $[\Delta t]$       & Betrachteter Zeitraum \dotfill & $\mathrm{s}$ \\
    $[W]$              & Umgewandelte elektrische Arbeit \dotfill & $\mathrm{J}$ \\
\end{tabular}


\subsubsection{Leistungsbeiwert}
Der Leistungsbeiwert beschreibt, wie viel der im Wind enthaltenen kinetischen Energie eine Windturbine maximal in nutzbare mechanische Leistung umwandeln kann. Laut Betz-Gesetz liegt das theoretische Maximum bei 0,593. In der Praxis erreichen moderne Windturbinen typischerweise Werte zwischen 0,4 und 0,5.

\subsubsection{Hohe Windgeschwindigkeiten}
\begin{itemize}
    \item Bei höheren Windgeschwindigkeit wird die Leistung begrenzt. Dies geschieht durch eine Pitch-Regelung (Blattwinkelverstellung).\\
    \item Abschaltung und ausdrehen aus dem Wind.\\
\end{itemize}

\vspace{-0.4cm}
\subsection{Netzkopplung}
DU = Direktumrichter, ZKU = Zwischenkreis-Umrichter\\
\vspace{-0.4cm}
\subsubsection{Direkte Netzkopplung mit ASM}
\includegraphics[width=0.5\columnwidth]{images/Direkte Netzkopplung mit ASM.png}
\vspace{-0.4cm}
\subsubsection{Direkte Netzkopplung mit SM}
\includegraphics[width=0.5\columnwidth]{images/Direkte Netzkopplung mit SM.png}
\vspace{-0.4cm}
\subsubsection{Direkte Netzkopplung mit ASM und DU im Läufer}
\includegraphics[width=0.5\columnwidth]{images/Direkte Netzkopplung mit ASM und Direktumrichter im Läufer.png}
\vspace{-0.4cm}
\subsubsection{Direkte Netzkopplung mit SM über Gleichstromzwischenkreis}
\begin{itemize}
    \item variable Drehzahl
    \item Verwendung Offshore wegen Leitungskapazitäten
\end{itemize}
\includegraphics[width=0.5\columnwidth]{images/Direkte Netzkopplung mit SM über einen Gleichstromzwischenkreis, variable Drehzahl.png}
\vspace{-0.2cm}
\subsubsection{Direkte Netzkopplung mit ASM und ZKU im Läufer}
\begin{minipage}[ht]{0.5\columnwidth}
    \begin{itemize}
    \item übersynchrone Stromrichter-Kaskade
    \end{itemize}
\end{minipage}
\begin{minipage}[ht]{0.5\columnwidth}
    \begin{itemize}
    \item variable Drehzahl
    \end{itemize}
\end{minipage}
\includegraphics[width=0.6\columnwidth]{images/Direkte Netzkopplung mit ASM und Zwischenkreis-umrichter im Läufer.png}

\newcolumn